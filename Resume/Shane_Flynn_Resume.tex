%=================20==================40=================60===================80
%                          User Packages/Definitions
%==============================================================================%
%==============================================================================%
%to do place education at the bottom. place projects at the top then skills
%then education
%have linked in github email and phone for contact info.
%==============================================================================%
%==============================================================================%
\documentclass[letterpaper]{article}
\usepackage{hyperref}
\usepackage{geometry}
%==============================================================================%
\def\name{Shane W. Flynn}
\def\footerlink{}

%==============================================================================%
%                                pdf metadata
%==============================================================================%
\hypersetup{
  colorlinks = true,
  urlcolor = black,
  pdfauthor = {\name},
  pdfkeywords = {},
  pdftitle = {\name: Curriculum Vitae},
  pdfsubject = {Curriculum Vitae},
  pdfpagemode = UseNone
}
%==============================================================================%
\geometry{
  body={6.5in, 10in},
  left=1.0in,
  top=0.5in
}
%==============================================================================%
% page headers
%==============================================================================%
\pagestyle{myheadings}
\markright{\name}
\thispagestyle{empty}

%==============================================================================%
% section fonts
%==============================================================================%
\usepackage{sectsty}
\sectionfont{\rmfamily\bfseries\large}
\subsectionfont{\rmfamily\bfseries}

% don't indent paragraphs.
\setlength\parindent{0pt}

% make lists without bullets
%\renewenvironment{itemize}{
%  \begin{list}{}{
%    \setlength{\leftmargin}{1.5em}
%  }
%}{
%  \end{list}
%}

\begin{document}

% Print name centered and bold:
\centerline{\huge \textbf{\name}}

\vspace{0.1in}

California Institute of Technology\hfill zhhuang@caltech.edu\\
Medical Engineering \hfill may.ziyuhuang@gmail.com \\
1200 E. California Blvd. MC 136-93\hfill 626-524-4881\\
Pasadena, CA 91125\hfill LinkedIn: www.linkedin.com/in/zi-yu-may-huang-03434bb1/\\
\hrule

%\section*{Career Objective}
%Medical engineering graduate student with a mechanical engineering background and laboratory research skills. Experience collaborating with medical physicians and pharmaceutical company to solve technical problems related to the medical field. Looking for a full-time position with an organization to solve unmet clinical needs. I combine my education as an engineer with my primary knowledge of design, micro/conventional fabrication, and medical devices to provide technical insight to research problems.

\vspace{0.1in}
%\section*{Education}
{\large \textbf{\underline{Education}}}\\
\vspace{-0.1in}\\
\textbf{Ph.D  Theoretical Chemistry}, University of California Irvine, Irvine,
CA, USA  \hfill 2017-2021\\
Research Advisor: Vladimir A. Mandelshtam \null \hfill (Expected)
\vspace{0.08in}

\textbf{M.S. Theoretical Chemistry}, California Institute of Technology, Pasadena, CA, USA  \hfill 2015-2017 \\
Research Advisor: William A. Goddard III\\
%GPA: 3.7/4.3
\vspace{0.08in}

\textbf{B.S. Chemistry, B.S. Biology, minor: Mathematics}, University of
Massachusetts, Boston, Boston MA \hfill 2010-2015\\
Research Advisors: Jason R. Green, Steven A. Ackermann\\
GPA: 3.7/4.0
\vspace{0.1in}

%\section*{Skills}
{\large \textbf{\underline{Skills}}}\\
\vspace{-0.1in}\\
\textbf{Programming Languages}: Python, Fortran, Java\\
%\textbf{Technical Skill}: Cleanroom fabrication (chemical vapor deposition, photolithography, plasma etching, etc.), 3D printing, cell cultural, genetic engineering basic techniques (DNA extraction, gene modification, gene transformation).\\
\vspace{-0.08in}

% \section*{Teaching Experience}
% \textbf{Teaching Assistant}\hfill 2017\\
% California Institute of Technology\\
% Professor: Yu-Chong Tai, Hyuck Choo\\
% Course: Principles and Design of Medical Devices (MedE201B)

%\section*{Research Projects}
%%%%{\large \textbf{\underline{Research Experience}}}\\
%%%%\vspace{-0.06in}\\
%%%%\textbf{Electrochemical impedance spectroscopy balloon catheter for plaque detection} \hfill Jan 2019-Present
%%%%\begin{itemize}
%%%%    \item Developed and fabricated an electrochemical impedance spectroscopy balloon catheter to improve current plaque detection procedures.
%%%%    \vspace{-0.05in}
%%%%    \item Collaborated with surgeons and MD/PhD students at UCLA for animal and human experiments.
%%%%    %\item Developed impedance 3D map to visualize the fat content distribution in blood vessel and conducted 3D simulation model in Comsol.
%%%%    \vspace{-0.05in}
%%%%    \item Improved 3D impedance mapping for visualizing the fat distribution inside blood vessels and developed 3D finite element simulations for these blood vessels in COMSOL.
%%%%    \vspace{-0.05in}
%%%%    \item Manuscript: Y. Luo, \textbf{Z.-Y. Huang}, P. Abiri, \textit{et al}. "Electrochemical Impedance Spectroscopic characterization  of Arterial Plaque using a dual-sensor system in Carotid Ligation and High-Fat-Induced Lesions." (in prep. 2020).
%%%%\end{itemize}
%%%%%Developed and fabricated a 6-point electrochemical impedance spectroscopy electrode on a balloon catheter. The device is capable of identifying vulnerable plaque and has been tested in live pig models, human explanted hearts, and human cadaver coronary arteries. Electrochemical impedance simulation in 3D reconstructed models based on the blood vessel histology has been conducted.
%%%%
%%%%\textbf{Electrochemical impedance tomography for human fatty liver quantification} \hfill June 2019-Present
%%%%\begin{itemize}
%%%%    %\item Improving fat content calculation algorithm from 2D model to 3D model in Matlab.
%%%%    \item Improved 2D fat content calculations and implemented a 3D algorithm using MATLAB.
%%%%    \vspace{-0.05in}
%%%%    \item Collaborated with PhD students at UCLA for human data collection and MRI scanning.
%%%%    \vspace{-0.05in}
%%%%    \item Manuscript: Y. Luo, \textbf{Z.-Y. Huang}, C.-C. Chang, \textit{et al}. "Non-Invasive Electrical Impedance Tomography to Quantify Human Fatty Infiltration." (in prep. 2020).
%%%%\end{itemize}
%%%%
%%%%\textbf{Optimization of solid lubricant coating for medical syringes} \hfill Feb 2017-Present
%%%%\begin{itemize}
%%%%    \item Designed and conducted a systematic study on the frictional properties of parylene as a biocompatible solid lubricant.
%%%%    \vspace{-0.05in}
%%%%    \item Collaborated with Amgen engineers to optimize the parylene coating parameters to enhance the performance of their current needle shields.
%%%%    \vspace{-0.05in}
%%%%    \item Developed a non-destructive method for needle shield coating to improve the quality control. %is this a full sentence?
%%%%\end{itemize}
%%%%%Conducted a systematic study on the static and kinetic friction coefficients as a function of thickness for parylene C and HT (a type of biocompatible polymer that is used as a solid lubricant for medical syringe rubber needle shields). Tests were performed on various types of medical syringe materials.
%%%%%The optimization of the parylene coating parameters allowed Amgen to enhance the performance of their autoinjectors.
%%%%%Developed an acrylic test-strip method for measuring parylene thickness inside a needle shield.
%%%%
%%%%\textbf{Local drug delivery catheter with blood bypassing functionality}
%%%%\hfill March 2018-Sept 2018
%%%%\begin{itemize}
%%%%    \item Developed and fabricated a double-ballooned local drug delivery catheter with blood bypassing capability that allows for extended use without causing downstream ischemia.
%%%%    \vspace{-0.05in}
%%%%    \item Designed and implemented prototype testing for functionality validation.
%%%%    \vspace{-0.05in}
%%%%    \item Collaborated with surgeons at UCLA for \textit{in vivo} pig experiments.
%%%%    \vspace{-0.05in}
%%%%    \item Presentation: “Double-Ballooned Local Drug Delivery Catheter with Blood Bypassing Function.” 20th Transducers Conference. June 2019. Berlin, Germany.
%%%%\end{itemize}
%%%%%Developed . This catheter is intended for high-dose local drug treatment of cardiovascular diseases such as atherosclerosis and restenosis. It allows for extended use without causing downstream ischemia. The functionality of the catheter was validated in both benchtop and live pig experiments.
%%%%
%%%%%\subsection*{Research (Select)}
%%%%%\subsection*{Publications/Presentations (Select)}
%%%%
%%%%%Presenter: “Double-ballooned Local Drug Delivery Catheter with Blood Bypassing Function.” June, 2019, Poster at the 20th Transducers. Berlin, Germany.
%%%%%\textbf{Z.-Y. Huang}, Y. Luo, P. Abiri, R. S. Packard, T. K. Hsiai, and Y.-C. Tai.
%%%%
%%%%
%%%%%Presenter: “Microenvironment of Cell Cultures on Digital Microfluidic Platforms.” Dec., 2014, Poster at the 31th National Conference on Mechanical Engineering of Chinese Society of Mechanical Engineers (CSME). Taichung, Taiwan.
%%%%%Z.-Y. Huang, S.-K. Fan.
%%%%
%%%%
%%%%%Co-Author: Y. Luo, \textbf{Z.-Y. Huang}, P. Abiri, \textit{et al}. "Electrochemical Impedance Spectroscopic characterization  of Arterial Plaque using a dual-sensor system in Carotid Ligation and High-Fat-Induced Lesions." (Manuscript in prep.)
%%%%%, R. S. Packard, T. K. Hsiai, and Y.-C. Tai.
%%%%
%%%%
%%%%%Co-Author: Y. Luo, \textbf{Z.-Y. Huang}, C.-C. Chang, \textit{et al}. "Non-Invasive Electrical Impedance Tomography to Quantify Human Fatty Infiltration." (Manuscript in prep.)
%%%%%, T. K. Hsiai, and Y.-C. Tai.
%%%%
%%%%%\vspace{0.1in}
%%%%%\hrule
%%%%
%%%%
%%%%%\hrule
%%%%
%%%%%\vspace{0.1in}
%%%%%\hrule
%%%%
%%%%% \section*{Awards and Honors (Select)}
%%%%
%%%%% \begin{itemize}
%%%%
%%%%% \item \textbf{Taiwan/Caltech Ministry of Education Fellowship}\\
%%%%% Ministry of Education, Taiwan\hfill 2015-2019
%%%%
%%%%% \item \textbf{Benjamin M. Rosen Graduate Fellowship}\\
%%%%% Caltech, California \hfill 2015-2016
%%%%
%%%%
%%%%% \item \textbf{College Student Research Fellowship}\\
%%%%% Project: Microenvironment with Growth Factor Gradient for Cell Culture on Digital Microfluidic Platform\\
%%%%% Ministry of Science and Technology, Taiwan\hfill 2014-2015
%%%%
%%%%% \item \textbf{Presidential Award}\\
%%%%% Obtained 7 times\\
%%%%% National Taiwan University, Taipei, Taiwan\hfill 2011-2015
%%%%
%%%%% \item \textbf{Best Group Award}\\
%%%%% Applied Materials Innovation Camp, Applied Materials Company\hfill 2013
%%%%
%%%%% \item \textbf{Best Poster Award}\\
%%%%% Poster: The Yeast Eating Woods\\
%%%%% BioSensing BioActuation BioNanotechnology Summer Institute, UIUC\hfill 2012
%%%%
%%%%
%%%%% \end{itemize}
%%%%
\end{document}
