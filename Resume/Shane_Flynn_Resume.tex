%=================20==================40=================60===================80
%                          User Packages/Definitions
%==============================================================================%
\documentclass[letterpaper]{article}
\usepackage{hyperref}
\usepackage{geometry}
%==============================================================================%
\def\name{Shane W. Flynn}
\def\footerlink{}
%==============================================================================%
%                                PDF Metadata
%==============================================================================%
\hypersetup{
  colorlinks = true,
  urlcolor = black,
  pdfauthor = {\name},
  pdfkeywords = {},
  pdftitle = {\name: Curriculum Vitae},
  pdfsubject = {Curriculum Vitae},
  pdfpagemode = UseNone
}
%==============================================================================%
\geometry{
  body={6.5in, 10in},
  left=1.0in,
  top=0.5in
}
%==============================================================================%
%                                Page Headers
%==============================================================================%
\pagestyle{myheadings}
\markright{\name}
\thispagestyle{empty}
%==============================================================================%
%                               Section Fonts
%==============================================================================%
\usepackage{sectsty}
\sectionfont{\rmfamily\bfseries\large}
\subsectionfont{\rmfamily\bfseries}
\setlength\parindent{0pt}
%==============================================================================%
\begin{document}
%==============================================================================%
\centerline{\huge \textbf{\name}}
\vspace{0.1in}
shane.flynn001@gmail.com \hfill (413)-841-5470\\
swflynn@uci.edu \hfill LinkedIn: \url{https://www.linkedin.com/in/shane-flynn-676b72184/}
\hrule
%==============================================================================%
%                                  Education
%==============================================================================%
\vspace{0.15in}
{\large \textbf{\underline{Education}}}\\
\vspace{-0.2in}\\

\textbf{Ph.D  Theoretical Chemistry}, University of California Irvine.
Irvine, CA, USA  \hfill 2017-2021\\
Research Advisor: Vladimir A. Mandelshtam \null \hfill (Expected)\\
\vspace{-0.1in}

\textbf{M.S. Theoretical Chemistry}, California Institute of Technology.
Pasadena, CA, USA  \hfill 2015-2017 \\
Research Advisor: William A. Goddard III\\
%GPA: 3.7/4.3
\vspace{-0.1in}

%Minor Mathematics
\textbf{B.S. Chemistry. B.S. Biology.}, University of
Massachusetts Boston.
Boston, MA, USA  \hfill 2010-2015 \\
Research Advisors: Jason R. Green. Steven M. Ackermann\\
%GPA: 3.7/4.0
%==============================================================================%
%                                  Skills/Tools
%==============================================================================%
%
%{\large \textbf{\underline{Skills}}}\\
%\vspace{-0.1in}\\
%Fortran, Python, C++, Git\\
%==============================================================================%
%                                  Research
%==============================================================================%

{\large \textbf{\underline{Research Experience}}}\\
\vspace{-0.06in}\\
\textbf{Quasi-Regular sampling of any distribution function} \hfill Jan 2017-Present
\begin{itemize}
    \item Derived and implemented a general sampling method generating points
    that are locally uniform, and sample the global distribution.
    \vspace{-0.05in}
    \item Derived and implemented a Distributed Gaussian Basis to compute
    the RoVibrational spectra for a given chemical system.
    \vspace{-0.05in}
    \item Quantitatively demonstrated the superior efficiency of QRGs compared
    to known methods such as metropolis Monte Carlo, and quasi-Monte Carlo.
\end{itemize}

\textbf{Thermodynamic analysis of polymer electrolytes for battery applications}
\hfill 2015-2017
\begin{itemize}
    \item Developed a computational screening paradigm to search for potential
    polymer electrolytes for battery applications.
    \vspace{-0.05in}
    \item Extended the use of the Two-Phase Thermodynamic model to study
    polymer thermodynamics independent of potential ions, greatly reducing the
    complexity associated with candidate screening.
\end{itemize}

\textbf{Quantifying disorder present in irreversibly decaying chemical
processes}
\hfill 2012-2015
\begin{itemize}
    \item Developed a new framework in chemical kinetics, resulting in
    a quantitative measurement for the cumulative fluctuations that occur in rate
    coefficients.
    \vspace{-0.05in}
    \item Derived a relationship between the rate coefficients found in chemical
    kinetics to the Fisher Information from information theory.
\end{itemize}
%==============================================================================%
%                                 Publications
%==============================================================================%
{\large \textbf{\underline{Publications}}}\\
\vspace{-0.3in}\\
\begin{itemize}
    \item \textbf{Shane W. Flynn}, Vladimir Mandelshtam.
    "Sampling general distributions with quasi-regular grids: Application to the
    vibrational spectra calculations".
    \vspace{-0.05in}
    \item Jonathan W. Nichols, \textbf{Shane W. Flynn}, Jason R.
    Green. "Order and disorder in irreversible decay processes".
    \vspace{-0.05in}
    \item \textbf{Shane W. Flynn}, Helen C. Zhao, Jason R. Green.
    "Measuring disorder in irreversible decay processes".
\end{itemize}
%==============================================================================%
%                                 Publications
%==============================================================================%
{\large \textbf{\underline{Teaching Experience (Teaching Assistant)}}}\\
\vspace{-0.3in}\\
\begin{itemize}
    \item \textbf{Mathematical Methods in Chemistry} \hfill 2020\\
    Graduate Level,Ch.237. University of California, Irvine.
    \vspace{-0.05in}
    \item \textbf{Thermodynamics and Introduction to Statistical Mechanics
    }\hfill 2019\\
    Graduate Level,Ch.232A. University of California, Irvine.
    \vspace{-0.05in}
    \item \textbf{Nature of the Chemical Bond}\hfill 2016\\
    Graduate Level, Ch.120A. California Institute of Technology, Pasadena.
    \vspace{-0.05in}
    \item \textbf{Linear Algebra} \hfill 2014\\
    Undergraduate Level, Ma.260. University of Massachusetts, Boston.
\end{itemize}


\end{document}
